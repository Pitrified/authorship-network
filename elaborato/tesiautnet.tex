%&TeX spellcheck = it_IT
\documentclass[12pt,a4paper,twoside]{report}
\usepackage{geometry}%to cus­tomize page lay­out
\usepackage[italian]{babel}%man­ages cul­tur­ally-de­ter­mined ty­po­graph­i­cal (and other) rules, and hy­phen­ation pat­terns
\usepackage[T1]{fontenc}%perchè faccia per bene i caratteri accentati nell'output
\usepackage[utf8]{inputenc}%perchè riconosca i caratteri accentati nel sorgente tex
\usepackage[hyphens]{url}%per inserire url
\usepackage{fancyhdr}%per header e foooter
\usepackage{graphicx}%per le immagini
\usepackage{wrapfig}%to wrap images with text
%\usepackage{epsfig}
\usepackage{amsthm}
\usepackage{listings}%per inserie codice sorgente
\usepackage{xcolor}%colori! :D
\usepackage{amsmath}%per formule matematiche
\usepackage{amssymb} %simboli matematici
\usepackage{booktabs}%en­hances the qual­ity of ta­bles
\usepackage{caption}%to cus­tomise the cap­tions in float­ing en­vi­ron­ments like fig­ure and ta­ble
\usepackage{subfig}%ma­nip­u­la­tion and ref­er­ence of small or ‘sub’ fig­ures and ta­bles within a sin­gle fig­ure or ta­ble en­vi­ron­ment
\usepackage{eurosym}%The Euro­pean cur­rency sym­bol for the Euro
\usepackage{siunitx}%tante cose sulle unità di misura
\usepackage{calligra}%cal­li­graphic font
\usepackage{verbatim}
\usepackage{indentfirst}%Indent first paragraph after section header
\usepackage{microtype}%fa diventare i font più belli
\usepackage{lmodern}%fontbelli
\usepackage{braket}%parentesi
\usepackage{mathtools}%to en­hance the ap­pear­ance of doc­u­ments con­tain­ing a lot of math­e­mat­ics
\usepackage{mathrsfs}  %per fare le lettere calligrafiche
\usepackage{bm}%define \bm{} that makes the argument bold
\usepackage{empheq}%markup per le formule di amsmath
\usepackage{pstricks}%macros for gen­er­at­ing PostScript that is us­able with most TeX macro for­mats
\usepackage{emptypage}%toglie il numero di pagina alle pahine create con clearpage e cleardoublepage
\usepackage{todonotes}
\usepackage{textcomp}%per il simbolo dei gradi
\usepackage{enumitem}%per controllare gli spazi nelle liste

%fine pacchetti

%:::::::::::::::::::::IMPOSTAZIONI::::::::::::::::::::::::::::::::::::::::::::::::::::::::::::::::::::

\geometry{a4paper,top=85pt,bottom=85pt,left=127pt,right=127pt} %imposto la geometria del documento
\captionsetup{tableposition=top,figureposition=bottom,font=small}  %impostazione dei caption di immagini e tabelle

%::::::::::::::::::::::::::::::::::::::::::::::::::::::::::::::::::::::::::::::::::::::::::::::::::::


\begin{document}

%::::::::::::::::::::::::::DEFINIZIONI DI COMANDI::::::::::::::::::::::::::::::::::::::::::::::::::::
\newcommand{\whitePage}{\cleardoublepage} % pagina bianca e rende la successiva una pagina destra

\newcommand{\HRule}{\rule{\linewidth}{0.6mm}} % command for the horizontal lines, change thickness here

\newcommand*{\eupgrave}{\MakeUppercase{è}} % maiuscola è accentata
%fine comandi
%::::::::::::::::::::::::::::::::::::::::::::::::::::::::::::::::::::::::::::::::::::::::::::::::::::



%::::::::::::::::::::::::::IMPOSTAZIONI GLOBALI::::::::::::::::::::::::::::::::::::::::::::::::::::::
% \setlist(nosep)
%::::::::::::::::::::::::::::::::::::::::::::::::::::::::::::::::::::::::::::::::::::::::::::::::::::


%::::::::::::::::::::::::::::::::TITLE PAGE::::::::::::::::::::::::::::::::::::::::::::::::::::::::::
\begin{titlepage}



\center  % Center everything on the page

\textsc{\Large \\[4.5cm]Università degli Studi di Padova}\\[1.5cm]
\textsc{\Large Corso di laurea triennale in Ingegneria dell'Informazione}\\[0.5cm]

\noindent\rule{\textwidth}{0.6pt}\\[0.4cm]
{ \Large{\bfseries Analisi di una rete di autori di pubblicazioni scientifiche}}\\[0.4cm]
\noindent\rule{\textwidth}{0.6pt}\\[1cm]

\includegraphics[height=4cm]{img/unipd-bn}\\[1cm]


\begin{minipage}{0.4\textwidth}
\begin{flushleft} \large
\emph{Laureando:}\\
Pietro Maria \textsc{Nobili} % Your name
\end{flushleft}
\begin{flushleft} \large
\emph{Matricola:}\\
1067941 % Your number
\end{flushleft}
\end{minipage}
~
\begin{minipage}{0.5\textwidth}
\begin{flushright} \large
\emph{Relatore:} \\
Prof.ssa Cinzia \textsc{Pizzi} \\% Supervisor's Name
\end{flushright}
\begin{flushright} \large
\emph{Correlatore:} \\
Dott. Mattia \textsc{Samory} % Supervisor's Name
\end{flushright}
\end{minipage}\\[2cm]


{\large 18 Luglio 2018}\\
{\large Anno Accademico 2017/2018}\\[3cm] % Date, change the \today to a set date if you want to be precise


\vfill % Fill the rest of the page with whitespace

\end{titlepage}
%::::::::::::::::::::::::::::::::::::::::::::::::::::::::::::::::::::::::::::::::::::::::::::::::::::


\whitePage

%::::::::::::::::::::::::::::::::::INDEX:::::::::::::::::::::::::::::::::::::::::::::::::::::::::::::

\pagenumbering{Roman}
\tableofcontents

%::::::::::::::::::::::::::::::::::::::::::::::::::::::::::::::::::::::::::::::::::::::::::::::::::::



%::::::::::::::::::::::::::::::::::THE REAL CONTENT::::::::::::::::::::::::::::::::::::::::::::::::::

\whitePage



\newpage
\pagenumbering{arabic}
\setcounter{page}{1}
\section*{Abstract} \label{abstract} % the asterisk is to remove the numbering
\addcontentsline{toc}{chapter}{Abstract} % we put the asterisk so we add it manually to the toc
% qua ci sara' un glorioso abstract
È stato generato un grafo delle pubblicazioni del DEI.

Sono stati cercati cluster nel grafo.

% TODO spezzettare le parole in un abstract è orribile
Sono stati confrontati con la struttura delle comunità del dipartimento.


% \whitePage
\section*{Introduzione} \label{introduzione}
\addcontentsline{toc}{chapter}{Introduzione} % we put the asterisk so we add it manually to the toc
Breve descrizione del community detection e della sua importanza, in generale e nel caso particolare delle comunità di autori di pubblicazioni scientifiche.

Presentazione della struttura della tesi.



\whitePage
\chapter{Community detection} \label{cap:comdet}

\section{Studi precedenti} \label{sec:storia}
Descrizione della community detection.

Descrizione della bibliografia attuale sui grafi di coautori.

\section{Metodi usati} \label{sec:metodi}
Che metodi vengono usati per generare le comunità.

\subsection{Girvan–Newman} \label{subsec:gn}
Descrizione metodo GN.

\subsection{Blockmodel} \label{subsec:bn}
Descrizione metodo blockmodel.

\subsection{Clauset-Newman-Moore} \label{subsec:cnm}
Descrizione metodo Clauset-Newman-Moore se implementato.



\whitePage
\chapter{Estrazione dati} \label{cap:estrazione}

\section{Struttura database microsoft} \label{sec:msr}
Il database è formato da vari file in plain-text che contengono i record di Autori, Paper e Affiliation.

\section{Processo di estrazione} \label{sec:processo}
%\subsection{Approccio iniziale} \label{subsec:primo}

Parto da una lista di nomi, in parte etichettati.

Estraggo gli ID autori.

Estraggo terne ID paper-autore-affiliation.

Creo gli edge.

Questo è un grafo, ma ha dei difetti:
\begin{itemize}[noitemsep, topsep=0pt]
\item
Vengono estratti gli ID di autori omonimi
\item
Allo stesso autore (una persona fisica) sono accoppiati più ID autore
\end{itemize}

\subsection{Filtro di autori per affiliation} \label{ssc:padovani}
Per risolvere il primo problema:

Il set di ID autori estratto viene filtrato, tenendo solo gli ID autore che hanno almeno un paper con affiliation padovana.

Scarto quelli che non hanno mai affiliation padovana.

Molti autori spuri vengono eliminati.

\subsection{Unione di ID autore in singoli nodi} \label{ssc:collassa}
Per risolvere il secondo problema, si propongono due modi:
\subsubsection{Per nome} \label{ssc:nomi}
I nodi con nomi uguali o abbreviazioni l'uno dell'altro sono considerati un unico nodo.


\subsubsection{Per distanza} \label{ssc:nomi}
Il metodo per nomi
%taggalo
introduce un errore potenzialmente molto grave, nomi come Michele Zorzi e Mattia Zorzi vengono confusi e considerati un unico nodo. Nella lista di nomi considerata questo succede solo nel caso citato, ma in dataset più ampi i falsi positivi aumentano considerevolmente. Si tenta di risolvere questo problema considerando unici due nodi in base alla distanza minima che hanno nel grafo.

I nodi con nomi uguali o abbreviazioni l'uno dell'altro sono considerati un unico nodo solo se sono anche vicini nel grafo: si calcola il cammino minimo tra i nodi e li si unisce se è di lunghezza minore di x

TODO Il valore ottimo di x sarebbe bello scoprirlo sperimentalmente


%\subsection{Secondo approccio} \label{subsec:secondo}


\whitePage
\chapter{Troubleshooting} \label{cap:trouble}
Il grafo generato presenta ancora delle carenze.

La lista di partenza include gli afferenti DEI attuali, mentre il database è del 2015. Questo comporta la mancanza di professori, non più a Padova, che avevano ruolo di aggregatore di una comunità.

TODO inserire il nome di Apostolico nella lista di partenza e valutare i cambiamenti

Un modo proposto per includere i nomi mancanti è: estrarre gli ID autore; estrarre i paper-aut-aff; da questa lista di paper estrarre tutti gli ID autore; potenzialmente ridurre i paper (e gli autori) estratti in base alle affiliation (anche solo del DEI e non di tutta Padova); iterare il processo con la nuova lista di autori.

In questo modo si estraggono anche troppe comunità, una singola collaborazione con un dipartimento esterno comporta alle iterazioni successive l'inclusione di molti autori di quel dipartimento. Un modo per risolvere il problema è, alla fine delle iterazioni e della creazione dei cluster, considerare solo quelli che contengono almeno un nome che era nella lista originale: in questo modo dipartimenti esterni, anche se connessi al grafo, non vengono inclusi.



\whitePage
\chapter{Risultati} \label{cap:risultati}
Descrizione della v-measure

Presentazione dei valori di v-measure per i vari grafi e commenti.



\whitePage
\chapter{Conclusioni} \label{cap:conclusioni}
Partendo da una lista di nomi di un dipartimento si può estrarre un grafo che lo rappresenti? Le comunità generate rispecchiano quelle reali?

%%% OBBLIGATORIA:


%::::::::::::::::::::::::::::::::::BIBLIOGRAFIA::::::::::::::::::::::::::::::::::::::::::::::::::::::

\nocite{*} % visto che non ho citato niente serve per non offendere bibtex
\bibliographystyle{plainurl}
% sarebbe bello usare bibtex ma non funziona, o forse non lo so usare
% vedremo di imparare
\bibliography{tesiautnet} %%% nome file(s)


\end{document}

%::::::::::::::::::::::::::::::::::::::::::::::::::::::::::::::::::::::::::::::::::::::::::::::::::::
