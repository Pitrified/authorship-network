%&TeX spellcheck = it_IT
\documentclass[12pt,a4paper,twoside]{report}
\usepackage{geometry}%to cus­tomize page lay­out
\usepackage[italian]{babel}%man­ages cul­tur­ally-de­ter­mined ty­po­graph­i­cal (and other) rules, and hy­phen­ation pat­terns
\usepackage[T1]{fontenc}%perchè faccia per bene i caratteri accentati nell'output
\usepackage[utf8]{inputenc}%perchè riconosca i caratteri accentati nel sorgente tex
\usepackage[hyphens]{url}%per inserire url
\usepackage{fancyhdr}%per header e foooter
\usepackage{graphicx}%per le immagini
\usepackage{wrapfig}%to wrap images with text
%\usepackage{epsfig}
\usepackage{amsthm}
\usepackage{listings}%per inserie codice sorgente
\usepackage{xcolor}%colori! :D
\usepackage{amsmath}%per formule matematiche
\usepackage{amssymb} %simboli matematici
\usepackage{booktabs}%en­hances the qual­ity of ta­bles
\usepackage{caption}%to cus­tomise the cap­tions in float­ing en­vi­ron­ments like fig­ure and ta­ble
\usepackage{subfig}%ma­nip­u­la­tion and ref­er­ence of small or ‘sub’ fig­ures and ta­bles within a sin­gle fig­ure or ta­ble en­vi­ron­ment
\usepackage{eurosym}%The Euro­pean cur­rency sym­bol for the Euro 
\usepackage{siunitx}%tante cose sulle unità di misura
\usepackage{calligra}%cal­li­graphic font
\usepackage{verbatim}
\usepackage{indentfirst}%Indent first paragraph after section header
\usepackage{microtype}%fa diventare i font più belli
\usepackage{lmodern}%fontbelli
\usepackage{braket}%parentesi
\usepackage{mathtools}%to en­hance the ap­pear­ance of doc­u­ments con­tain­ing a lot of math­e­mat­ics
\usepackage{mathrsfs}  %per fare le lettere calligrafiche
\usepackage{bm}%define \bm{} that makes the argument bold
\usepackage{empheq}%markup per le formule di amsmath
\usepackage{pstricks}%macros for gen­er­at­ing PostScript that is us­able with most TeX macro for­mats
\usepackage{emptypage}%toglie il numero di pagina alle pahine create con clearpage e cleardoublepage
\usepackage{todonotes}
\usepackage{textcomp}%per il simbolo dei gradi
\usepackage{enumitem}%per controllare gli spazi nelle liste

%fine pacchetti

%:::::::::::::::::::::IMPOSTAZIONI::::::::::::::::::::::::::::::::::::::::::::::::::::::::::::::::::::

\geometry{a4paper,top=85pt,bottom=85pt,left=127pt,right=127pt} %imposto la geometria del documento
\captionsetup{tableposition=top,figureposition=bottom,font=small}  %impostazione dei caption di immagini e tabelle

%::::::::::::::::::::::::::::::::::::::::::::::::::::::::::::::::::::::::::::::::::::::::::::::::::::


\begin{document}

%::::::::::::::::::::::::::DEFINIZIONI DI COMANDI::::::::::::::::::::::::::::::::::::::::::::::::::::
\newcommand{\whitePage}{\cleardoublepage}%fa una pagina bianca e fa in modo che la successiva sia un pagina destra

\newcommand{\HRule}{\rule{\linewidth}{0.6mm}} % Defines a new command for the horizontal lines, change thickness here
%fine comandi
%::::::::::::::::::::::::::::::::::::::::::::::::::::::::::::::::::::::::::::::::::::::::::::::::::::

%::::::::::::::::::::::::::::::::TITLE PAGE::::::::::::::::::::::::::::::::::::::::::::::::::::::::::
\begin{titlepage}



\center  % Center everything on the page

\textsc{\Large \\[4.5cm]Università degli Studi di Padova}\\[1.5cm]
\textsc{\Large Corso di laurea triennale in Ingegneria dell'Informazione}\\[0.5cm]

\noindent\rule{\textwidth}{0.6pt}\\[0.4cm]
{ \Large{\bfseries DEI authorship network}}\\[0.4cm]
\noindent\rule{\textwidth}{0.6pt}\\[1cm]

\includegraphics[height=4cm]{img/unipd-bn}\\[1cm]


\begin{minipage}{0.4\textwidth}
\begin{flushleft} \large
\emph{Laureando:}\\
Pietro Maria \textsc{Nobili} % Your name
\end{flushleft}
\begin{flushleft} \large
\emph{Matricola:}\\
1067941 % Your number
\end{flushleft}
\end{minipage}
~
\begin{minipage}{0.5\textwidth}
\begin{flushright} \large
\emph{Relatore:} \\
Prof. Cinzia \textsc{Pizzi} \\% Supervisor's Name
\emph{Correlatore:} \\
Dr. Mattia \textsc{Samory} % Supervisor's Name
\end{flushright}
\end{minipage}\\[2cm]


{\large 18 Luglio 2018}\\
{\large Anno Accademico 2017/2018}\\[3cm] % Date, change the \today to a set date if you want to be precise


\vfill % Fill the rest of the page with whitespace

\end{titlepage}
%::::::::::::::::::::::::::::::::::::::::::::::::::::::::::::::::::::::::::::::::::::::::::::::::::::


\whitePage

%::::::::::::::::::::::::::::::::::INDEX:::::::::::::::::::::::::::::::::::::::::::::::::::::::::::::

\pagenumbering{Roman}
\tableofcontents

%::::::::::::::::::::::::::::::::::::::::::::::::::::::::::::::::::::::::::::::::::::::::::::::::::::



%::::::::::::::::::::::::::::::::::THE REAL CONTENT::::::::::::::::::::::::::::::::::::::::::::::::::

\whitePage



\newpage
\pagenumbering{arabic}
\setcounter{page}{1}
\section*{Sommario} \label{sommario} % the asterisk is to remove the numbering
\addcontentsline{toc}{chapter}{Sommario} % we put the asterisk so we add it manually to the toc
% qua ci sara' un glorioso sommario


\whitePage
\chapter{Introduzione} \label{cap:introduzione}

\section{Gli autori - i grafi - gli studi gia' esistenti} \label{sec:storia}


\section{Database microsoft} \label{sec:msr}


\whitePage	
\chapter{Estrazione dati} \label{cap:estrazione}


\whitePage
\chapter{Risultati} \label{cap:risultati}


\chapter{Conclusioni} \label{cap:conclusioni}

%%% OBBLIGATORIA:


%::::::::::::::::::::::::::::::::::BIBLIOGRAFIA::::::::::::::::::::::::::::::::::::::::::::::::::::::

\nocite{*} % visto che non ho citato niente serve per non offendere bibtex
\bibliographystyle{plainurl}
% sarebbe bello usare bibtex ma non funziona, o forse non lo so usare
% vedremo di imparare
\bibliography{tesiautnet} %%% nome file(s)


\end{document}

%::::::::::::::::::::::::::::::::::::::::::::::::::::::::::::::::::::::::::::::::::::::::::::::::::::
